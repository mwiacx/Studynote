\documentclass[a4paper, 12pt]{report}
%import package
%use xelatex to compile please !
\usepackage{amsmath}
\usepackage{paralist}
\usepackage{listings}
\usepackage[svgnames, table]{xcolor}
\usepackage{indentfirst}
\usepackage[CJKchecksingle, CJKnumber]{xeCJK}
%\usepackage{metalogo}
\usepackage[colorlinks,linkcolor=blue]{hyperref}
\usepackage[rm]{titlesec}% 改变章节标题格式
\usepackage{geometry}
\usepackage{booktabs}
\usepackage{graphicx}
\usepackage{enumitem}

%code highlight
\lstset{
    language = C,
    basicstyle = \ttfamily \small,
    flexiblecolumns = false,
    tabsize = 4,
    breaklines = true,
    basewidth = {0.5em, 0.45em},
    boxpos = t,
    backgroundcolor = \color[RGB]{245,245,244},
    keywordstyle = \bf\color{blue},
    %identifierstyle = \bf,
    commentstyle = \color[RGB]{0,139,0},
    numberstyle = \color[RGB]{0,192,192},
    stringstyle = \color[RGB]{128,0,0},
    rulesepcolor = \color[RGB]{102,102,102},
    frame = shadowbox,
    numbers = left,
    numbersep = 7pt
}

%setup Chinese & English fonts
%...
%\setmainfont[Mapping=tex-text]{Times New Roman}
\setmainfont[Mapping=tex-text]{Adobe Garamond Pro}
\setmonofont{Courier 10 Pitch}
\setCJKmainfont[BoldFont={Adobe Heiti Std}, ItalicFont={Adobe Kaiti Std}]{Adobe Song Std}
\setCJKsansfont{Adobe Heiti Std}
\setCJKmonofont{Adobe Fangsong Std}

\punctstyle{hangmobanjiao}
\parindent 2em
\linespread{1.2}
\setlist[description]{leftmargin=\parindent,labelindent=\parindent}

%编号层次
\setcounter{secnumdepth}{3} 
\setcounter{tocdepth}{3} 

%fix "no-break space" character error
%\DeclareUnicodeCharacter{00A0}{~}

%重命名
\renewcommand {\contentsname }{目\qquad 录}
\renewcommand {\listfigurename }{图\ 目\ 录}
\renewcommand {\listtablename }{表\ 目\ 录}
\renewcommand {\figurename }{图}
\renewcommand {\tablename }{表}
\renewcommand {\bibname }{参\ 考\ 文\ 献}
\renewcommand{\equationautorefname}{公式}
\renewcommand{\footnoteautorefname}{脚注}
\renewcommand{\itemautorefname}{项}
\renewcommand{\figureautorefname}{图}
\renewcommand{\tableautorefname}{表}
\renewcommand{\appendixautorefname}{附录}
\renewcommand{\theoremautorefname}{定理}
%\titleformat{\part}[display]{\centering\Huge}{\textbf{第~\thepart~部分}}{0.2cm}{}
\titleformat{\chapter}[block]{\centering\Huge\bfseries}{\textbf{第~\thechapter~章}}{1em}{}
\titleformat{\section}[block]{\LARGE\bfseries}{\textbf{\thesection}}{0.6em}{}
\titleformat{\subsection}[block]{\large\bfseries}{\textbf{\thesubsection}}{0.4em}{}
%自定义交叉引用
\newcommand{\subsecref[1]}{~\ref{#1}~小节}
%\newcommand{\tableref[1]}{表~\ref{#1}~}

%页面边距
\newgeometry{
    top=25mm, bottom=25mm, left=30mm, right=20mm,
    headsep=5mm, headheight=10mm, footskip=10mm,
}
\savegeometry{mgeometry}
\loadgeometry{mgeometry}

\renewcommand{\baselinestretch}{1.5}
\setlength{\parindent}{2em}
\setlength{\floatsep}{3pt plus 3pt minus 2pt}      % 图形之间或图形与正文之间的距离
\setlength{\abovecaptionskip}{10pt plus 1pt minus 1pt} % 图形中的图与标题之间的距离
\setlength{\belowcaptionskip}{3pt plus 1pt minus 2pt} % 表格中的表与标题之间的距离

%标题页
%Original author: Peter Wilson (herries.press@earthlink.net)
\newcommand*{\plogo}{\fbox{$\mathcal{BUAALES}$}} % Generic publisher logo
\newcommand*{\titleAT}{\begingroup % Create the command for including the title page in the document
    \newlength{\drop} % Command for generating a specific amount of whitespace
    \drop=0.1\textheight % Define the command as 10% of the total text height
    
    \centering % Center all text
    \rule{\textwidth}{1pt}\par % Thick horizontal line
    \vspace{2pt}\vspace{-\baselineskip} % Whitespace between lines
    \rule{\textwidth}{0.4pt}\par % Thin horizontal line
    
    \vspace{\drop} % Whitespace between the top lines and title
    \textcolor{Red}{ % Red font color
        {\Huge 阅读笔记}\\[0.5\baselineskip] % Title line 1
        {\Large OF}\\[0.75\baselineskip] % Title line 2
        {\Huge Barrelfish}} % Title line 3
    
    \vspace{0.25\drop} % Whitespace between the title and short horizontal line
    \rule{0.3\textwidth}{0.4pt}\par % Short horizontal line under the title
    \vspace{\drop} % Whitespace between the thin horizontal line and the author name
    
    {\large \textsc{陈逊}}\par % Author name
    
    \vfill % Whitespace between the author name and publisher text
    {\large \textcolor{Blue}{\plogo}}\\[0.5\baselineskip] % Publisher logo
    {\large \textsc{北京航空航天大学}}\par % Publisher
    
    \vspace*{\drop} % Whitespace under the publisher text
    
    \rule{\textwidth}{0.4pt}\par % Thin horizontal line
    \vspace{2pt}\vspace{-\baselineskip} % Whitespace between lines
    \rule{\textwidth}{1pt}\par % Thick horizontal line
    
    \endgroup}

\begin{document}
    
    \pagestyle{empty} % Removes page numbers
    \titleAT % This command includes the title page
    %\maketitle
    %contents
    
    \tableofcontents
    %\listoffigures
    %\listoftables
	
	\chapter{capability}
	
	\section{Cscope笔记}
    
    cscope的基本操作:
    
    \begin{table}[htbp]
        \centering
        \begin{tabular}{cl}
            \toprule
            操作符 & 备注 \\
            \midrule
            s & 查找这个C符号 \\
            g & 查找这个定义 \\
            d & 查找被这个函数调用的函数(们) \\
            c & 查找调用这个函数的函数(们)\\
            t & 查找这个字符串 \\
            e & 查找这个egrep匹配模式 \\
            f & 找这个文件 \\
            i & 查找\#include这个文件的文件(们) \\
            \bottomrule
        \end{tabular}
    \end{table}
	
	\section{数据结构}
    
    capability权限位:
    
    \begin{table}[htbp]
        \centering
        \caption{capability权限位}
        \label{table:right}
        \begin{tabular}{lcr}
            \toprule
            名称 & 位段 & 备注 \\
            \midrule
            RBZ & 5...8 & 保留位 \\
            CAPRIGHTS\_IDENTIFY & 4 & 暂无 \\
            CAPRIGHTS\_GRANT & 3 & 暂无 \\
            CAPRIGHTS\_EXECUTE & 2 & 可执行 \\
            CAPRIGHTS\_WRITE & 1 & 可写 \\
            CAPRIGHTS\_READ & 0 & 可读 \\
            \bottomrule
           \end{tabular}
       \end{table} 
	
    capablity类型:
    
    \chapter{Kernel系统调用}
    
    \begin{table}[htbp]
        \centering
        \caption{ObjType\_Dispatcher}
        \begin{tabular}{lll}
            \toprule
            系统调用 & 接口函数 & 备注 \\
            \midrule
            DispatcherCmd\_Setup & handle\_dispatcher\_setup & 保留位 \\
            DispatcherCmd\_Properties & handle\_dispatcher\_properties & 暂无 \\
            DispatcherCmd\_PerfMon & handle\_dispatcher\_perfmon & 暂无 \\
            DispatcherCmd\_DumpPTables & dispatcher\_dump\_ptables & 可执行 \\
            DispatcherCmd\_DumpCapabilities & dispatcher\_dump\_capabilities & 可写 \\
            \bottomrule
        \end{tabular}
    \end{table} 
    
    \begin{table}[htbp]
        \centering
        \caption{ObjType\_KernelControlBlock}
        \begin{tabular}{lll}
            \toprule
            系统调用 & 接口函数 & 备注 \\
            \midrule
            FrameCmd\_Identify & handle\_kcb\_identify & 保留位 \\
            KCBCmd\_Clone & handle\_kcb\_clone & 暂无 \\
            \bottomrule
        \end{tabular}
    \end{table}
    
    \begin{table}[htbp]
        \centering
        \caption{ObjType\_Frame}
        \begin{tabular}{lll}
            \toprule
            系统调用 & 接口函数 & 备注 \\
            \midrule
            FrameCmd\_Identify & handle\_kcb\_handle\_frame\_identify & 保留位 \\
            \bottomrule
           \end{tabular}
    \end{table} 
    
    \begin{table}[htbp]
        \centering
        \caption{ObjType\_DevFrame}
        \begin{tabular}{lll}
            \toprule
            系统调用 & 接口函数 & 备注 \\
            \midrule
            FrameCmd\_Identify & handle\_kcb\_handle\_frame\_identify & 保留位 \\
            \bottomrule
        \end{tabular}
    \end{table} 
    
    \begin{table}[htbp]
        \centering
        \caption{ObjType\_L1CNode}
        \begin{tabular}{lll}
            \toprule
            系统调用 & 接口函数 & 备注 \\
            \midrule
            CNodeCmd\_Copy & handle\_copy & 保留位 \\
            CNodeCmd\_Mint & handle\_mint & \\
            CNodeCmd\_Retype & handle\_retype & \\
            CNodeCmd\_Delete & handle\_delete & \\
            CNodeCmd\_Revoke & handle\_revoke & \\
            CNodeCmd\_Create & handle\_create & \\
            CNodeCmd\_GetState & handle\_get\_state & \\
            CNodeCmd\_GetSize & handle\_get\_size & \\
            CNodeCmd\_Resize & handle\_resize & \\
            \bottomrule
        \end{tabular}
    \end{table} 
    
    \begin{table}[htbp]
        \centering
        \caption{ObjType\_L2CNode}
        \begin{tabular}{lll}
            \toprule
            系统调用 & 接口函数 & 备注 \\
            \midrule
            CNodeCmd\_Copy & handle\_copy & 保留位 \\
            CNodeCmd\_Mint & handle\_mint & \\
            CNodeCmd\_Retype & handle\_retype & \\
            CNodeCmd\_Delete & handle\_delete & \\
            CNodeCmd\_Revoke & handle\_revoke & \\
            CNodeCmd\_Create & handle\_create & \\
            CNodeCmd\_GetState & handle\_get\_state & \\
            CNodeCmd\_Resize & handle\_resize & \\
            \bottomrule
        \end{tabular}
    \end{table}
    
    \begin{table}[htbp]
        \centering
        \caption{ObjType\_VNode\_ARM\_l1}
        \begin{tabular}{lll}
            \toprule
            系统调用 & 接口函数 & 备注 \\
            \midrule
            VNodeCmd\_Identify & handle\_vnode\_identify & 保留位 \\
            VNodeCmd\_Map & handle\_map & \\
            VNodeCmd\_Unmap & handle\_unmap & \\
            \bottomrule
        \end{tabular}
    \end{table}
    
    \begin{table}[htbp]
        \centering
        \caption{ObjType\_VNode\_ARM\_l2}
        \begin{tabular}{lll}
            \toprule
            系统调用 & 接口函数 & 备注 \\
            \midrule
            VNodeCmd\_Identify & handle\_vnode\_identify & 保留位 \\
            VNodeCmd\_Map & handle\_map & \\
            VNodeCmd\_Unmap & handle\_unmap & \\
            \bottomrule
        \end{tabular}
    \end{table}
    
    \begin{table}[htbp]
        \centering
        \caption{ObjType\_Frame\_Mapping}
        \begin{tabular}{lll}
            \toprule
            系统调用 & 接口函数 & 备注 \\
            \midrule
            MappingCmd\_Destroy & handle\_mapping\_destroy & 保留位 \\
            MappingCmd\_Modify & handle\_mapping\_modify & \\
            \bottomrule
        \end{tabular}
    \end{table}
    
    \begin{table}[htbp]
        \centering
        \caption{ObjType\_DevFrame\_Mapping}
        \begin{tabular}{lll}
            \toprule
            系统调用 & 接口函数 & 备注 \\
            \midrule
            MappingCmd\_Destroy & handle\_mapping\_destroy & 保留位 \\
            MappingCmd\_Modify & handle\_mapping\_modify & \\
            \bottomrule
        \end{tabular}
    \end{table}
    
    \begin{table}[htbp]
        \centering
        \caption{ObjType\_VNode\_ARM\_l1\_Mapping}
        \begin{tabular}{lll}
            \toprule
            系统调用 & 接口函数 & 备注 \\
            \midrule
            MappingCmd\_Destroy & handle\_mapping\_destroy & 保留位 \\
            MappingCmd\_Modify & handle\_mapping\_modify & \\
            \bottomrule
        \end{tabular}
    \end{table}
    
    \begin{table}[htbp]
        \centering
        \caption{ObjType\_VNode\_ARM\_l2\_Mapping}
        \begin{tabular}{lll}
            \toprule
            系统调用 & 接口函数 & 备注 \\
            \midrule
            MappingCmd\_Destroy & handle\_mapping\_destroy & 保留位 \\
            MappingCmd\_Modify & handle\_mapping\_modify & \\
            \bottomrule
        \end{tabular}
    \end{table}
    
    \begin{table}[htbp]
        \centering
        \caption{ObjType\_IRQTable}
        \begin{tabular}{lll}
            \toprule
            系统调用 & 接口函数 & 备注 \\
            \midrule
            IRQTableCmd\_Set & handle\_irq\_table\_set & 保留位 \\
            IRQTableCmd\_Delete & handle\_irq\_table\_delete & \\
            \bottomrule
        \end{tabular}
    \end{table}
    
    \begin{table}[htbp]
        \centering
        \caption{ObjType\_IPI}
        \begin{tabular}{lll}
            \toprule
            系统调用 & 接口函数 & 备注 \\
            \midrule
            IPICmd\_Send\_Start & monitor\_spawn\_core & 保留位 \\
            \bottomrule
        \end{tabular}
    \end{table}
    
    \begin{table}[htbp]
        \centering
        \caption{ObjType\_ID}
        \begin{tabular}{lll}
            \toprule
            系统调用 & 接口函数 & 备注 \\
            \midrule
            IDCmd\_Identify & handle\_idcap\_identify & 保留位 \\
            \bottomrule
        \end{tabular}
    \end{table}
    
    \begin{table}[htbp]
        \centering
        \caption{ObjType\_Kernel}
        \begin{tabular}{lll}
            \toprule
            系统调用 & 接口函数 & 备注 \\
            \midrule
            KernelCmd\_Cap\_has\_relations & monitor\_cap\_has\_relations & 保留位 \\
            KernelCmd\_Clear\_step & monitor\_handle\_clear\_step & \\
            KernelCmd\_Copy\_existing & monitor\_copy\_existing & \\
            KernelCmd\_Create\_cap & monitor\_create\_cap & \\
            KernelCmd\_Delete\_foreigns & monitor\_handle\_delete\_foreigns & \\
            KernelCmd\_Delete\_last & monitor\_handle\_delete\_last & \\
            KernelCmd\_Delete\_step & monitor\_handle\_delete\_step & \\
            KernelCmd\_Domain\_Id & monitor\_handle\_domain\_id & \\
            KernelCmd\_Get\_arch\_id & monitor\_get\_arch\_id & \\
            KernelCmd\_Get\_cap\_owner & monitor\_get\_cap\_owner & \\
            KernelCmd\_Get\_core\_id & monitor\_get\_core\_id & \\
            KernelCmd\_Has\_descendants & monitor\_handle\_has\_descendants & \\
            KernelCmd\_Is\_retypeable & monitor\_handle\_is\_retypeable & \\
            KernelCmd\_Identify\_cap & monitor\_identify\_cap & \\
            KernelCmd\_Identify\_domains\_cap & monitor\_identify\_domains\_cap & \\
            KernelCmd\_Lock\_cap & monitor\_lock\_cap & \\
            KernelCmd\_Nullify\_cap & monitor\_nullify\_cap & \\
            KernelCmd\_Register & monitor\_handle\_register & \\
            KernelCmd\_Remote\_relations & monitor\_remote\_relations & \\
            KernelCmd\_Retype & monitor\_handle\_retype & \\
            KernelCmd\_Revoke\_mark\_relations & monitor\_handle\_revoke\_mark\_rels & \\
            KernelCmd\_Revoke\_mark\_target & monitor\_handle\_revoke\_mark\_tgt & \\
            KernelCmd\_Set\_cap\_owner & monitor\_set\_cap\_owner & \\
            KernelCmd\_Spawn\_core & monitor\_spawn\_core & \\
            KernelCmd\_Unlock\_cap & monitor\_unlock\_cap & \\
            KernelCmd\_Get\_platform & monitor\_get\_platform & \\
            \bottomrule
        \end{tabular}
    \end{table}
    
    \chapter{Kernel dispatcher调度}
    
    \section{kernel/dispath.c文件}
    
    dispatch函数主要逻辑:
    \begin{enumerate}
        \item 检查参数dcb是不是当前dcb,如果不是切换到参数dcb,包括上下文
        \item 更新disp的系统时间为systiem\_now + kernel\_off
        \item 根据disp的状态:disable执行resume,enable执行excute(调度线程)。
    \end{enumerate}
    
    统计:
    
    计算题:画图时间轴
    
    1. 有一种典型的误解:多道程序是对于程序来说,而分时系统是对多用户来说的。 
    
    2.演化过程不清楚,一个技术->发现缺点->新的技术,这个过程讲不清楚。
    
    3.区别并不是理解的全面。
    
    4.如何设计可移植的OS,答案5花8门,也不全,建议讲一下。
        5.1 片面理解微内核。
        5.2 嵌入式OS比通用OS好。
        5.3 好几个同学把CPU写出CUP。
        
    5.答案比较雷同。操作系统如何抽象处理机,内存,外设答得不好
    
    6.有些同学不写,或者写的特别简单,也不解释一下。
    
    
\end{document} 
