\documentclass[a4paper, 12pt]{report}
%import package
%use xelatex to compile please !
\usepackage{amsmath}
\usepackage{paralist}
\usepackage{listings}
\usepackage[svgnames, table]{xcolor}
\usepackage{indentfirst}
\usepackage[CJKchecksingle, CJKnumber]{xeCJK}
%\usepackage{metalogo}
\usepackage[colorlinks,linkcolor=blue]{hyperref}
\usepackage[rm]{titlesec}% 改变章节标题格式
\usepackage{geometry}
\usepackage{booktabs}
\usepackage{graphicx}
\usepackage{enumitem}

%code highlight
\lstset{
    language = C,
    basicstyle = \ttfamily \small,
    flexiblecolumns = false,
    tabsize = 4,
    breaklines = true,
    basewidth = {0.5em, 0.45em},
    boxpos = t,
    backgroundcolor = \color[RGB]{245,245,244},
    keywordstyle = \bf\color{blue},
    %identifierstyle = \bf,
    commentstyle = \color[RGB]{0,139,0},
    numberstyle = \color[RGB]{0,192,192},
    stringstyle = \color[RGB]{128,0,0},
    rulesepcolor = \color[RGB]{102,102,102},
    frame = shadowbox,
    numbers = left,
    numbersep = 7pt
}

%setup Chinese & English fonts
%...
%\setmainfont[Mapping=tex-text]{Times New Roman}
\setmainfont[Mapping=tex-text]{Adobe Garamond Pro}
\setmonofont{Courier 10 Pitch}
\setCJKmainfont[BoldFont={Adobe Heiti Std}, ItalicFont={Adobe Kaiti Std}]{Adobe Song Std}
\setCJKsansfont{Adobe Heiti Std}
\setCJKmonofont{Adobe Fangsong Std}

\punctstyle{hangmobanjiao}
\parindent 2em
\linespread{1.2}
\setlist[description]{leftmargin=\parindent,labelindent=\parindent}

%编号层次
\setcounter{secnumdepth}{3} 
\setcounter{tocdepth}{3} 

%fix "no-break space" character error
%\DeclareUnicodeCharacter{00A0}{~}

%重命名
\renewcommand {\contentsname }{目\qquad 录}
\renewcommand {\listfigurename }{图\ 目\ 录}
\renewcommand {\listtablename }{表\ 目\ 录}
\renewcommand {\figurename }{图}
\renewcommand {\tablename }{表}
\renewcommand {\bibname }{参\ 考\ 文\ 献}
\renewcommand{\equationautorefname}{公式}
\renewcommand{\footnoteautorefname}{脚注}
\renewcommand{\itemautorefname}{项}
\renewcommand{\figureautorefname}{图}
\renewcommand{\tableautorefname}{表}
\renewcommand{\appendixautorefname}{附录}
\renewcommand{\theoremautorefname}{定理}
%\titleformat{\part}[display]{\centering\Huge}{\textbf{第~\thepart~部分}}{0.2cm}{}
\titleformat{\chapter}[block]{\centering\Huge\bfseries}{\textbf{第~\thechapter~章}}{1em}{}
\titleformat{\section}[block]{\large\bfseries}{\textbf{\thesection}}{0.7em}{}
\titleformat{\subsection}[block]{\large\bfseries}{\textbf{\thesubsection}}{0.5em}{}
%自定义交叉引用
\newcommand{\secref[1]}{~\ref{#1}节}
\newcommand{\subsecref[1]}{~\ref{#1}小节}
%\newcommand{\tableref[1]}{表~\ref{#1}~}

%页面边距
\newgeometry{
    top=25mm, bottom=25mm, left=20mm, right=20mm,
    headsep=5mm, headheight=10mm, footskip=10mm,
}
\savegeometry{mgeometry}
\loadgeometry{mgeometry}

\renewcommand{\baselinestretch}{1.5}
\setlength{\parindent}{2em}
\setlength{\floatsep}{3pt plus 3pt minus 2pt}      % 图形之间或图形与正文之间的距离
\setlength{\abovecaptionskip}{10pt plus 1pt minus 1pt} % 图形中的图与标题之间的距离
\setlength{\belowcaptionskip}{3pt plus 1pt minus 2pt} % 表格中的表与标题之间的距离

%标题页
%Original author: Peter Wilson (herries.press@earthlink.net)
\newcommand*{\plogo}{\fbox{$\mathcal{BuaaLes}$}} % Generic publisher logo
\newcommand*{\titleAT}{\begingroup % Create the command for including the title page in the document
    \newlength{\drop} % Command for generating a specific amount of whitespace
    \drop=0.1\textheight % Define the command as 10% of the total text height
    
    \centering % Center all text
    \rule{\textwidth}{1pt}\par % Thick horizontal line
    \vspace{2pt}\vspace{-\baselineskip} % Whitespace between lines
    \rule{\textwidth}{0.4pt}\par % Thin horizontal line
    
    \vspace{\drop} % Whitespace between the top lines and title
    \textcolor{Red}{ % Red font color
        {\Huge Barrelfish验证报告}\\[0.5\baselineskip] % Title line 1
        {\Large 自裁剪和中断分发}} % Title line 3
    
    \vspace{0.25\drop} % Whitespace between the title and short horizontal line
    \rule{0.3\textwidth}{0.4pt}\par % Short horizontal line under the title
    \vspace{\drop} % Whitespace between the thin horizontal line and the author name
    
    {\Large 陈~~~~逊}\par % Author name
    {\Large 孙思杰}\par % Student ID
    
    \vfill % Whitespace between the author name and publisher text
    %%{\large \textcolor{Blue}{\plogo}}\\[0.5\baselineskip] % Publisher logo
    {\large \textsc{POWERED BY \LaTeX}}\par % Publisher
    
    \vspace*{\drop} % Whitespace under the publisher text
    
    \rule{\textwidth}{0.4pt}\par % Thin horizontal line
    \vspace{2pt}\vspace{-\baselineskip} % Whitespace between lines
    \rule{\textwidth}{1pt}\par % Thick horizontal line
    
    \endgroup}

\begin{document}
    
    \pagestyle{empty} % Removes page numbers
    \titleAT % This command includes the title page
    
    \newpage
    
    \tableofcontents
    
    \newpage
    
    \chapter{自裁剪论证}
    
    \section{Hakefile生成的目标文件类型}
    
    Hakefile文件定义了两种生成目标。一种是库,另一种是应用程序。
    
    \section{库的生成}
    
    在Barrelfish源码树中的lib文件夹中,包含所有可能用到的库。
    
    \begin{figure}[htbp]
        \centering
        \includegraphics[scale=0.75]{./img1.png}
        \caption{lib列表}
        \label{fig:lib_img}
    \end{figure}
    
    在每个库的根目录下,有一个Hakefile,定义了生成的Library目标文件以及该Library需要的所有C文件。
    
    \begin{lstlisting}
    build library 
    { 
        target = "vfs",
        cFiles = [ "vfs.c", "vfs_path.c", "fopen.c", "mmap.c"],
        mackerelDevices = [ "ata_identify", "fat_bpb", "fat16_ebpb"],
        flounderBindings = [ "trivfs", "bcache", "ahci_mgmt", "ata_rw28" ],
        flounderExtraBindings=[("trivfs",["rpcclient"]), ("bcache", ["rpcclient"])],
        flounderDefs = [ "monitor" ]
    },
    \end{lstlisting}
    
    在由此Hakefile生成的Makefile对应的规则中,生成的目标文件是<库名称>.a。库的所有c文件都会分别生成一个.o文件,所有的.o文件打包在一起成为了.a文件。
    
    \section{应用程序的生成}
    
    此处以Barrelfish提供的fish程序为例进行论证。

    在fish程序源码根目录中的Hakefile文件中有如下声明。
    
    \begin{lstlisting}
    build application 
    { 
        target = "fish",
        cFiles = [ "fish_common.c", "fish_arm.c", "font.c" ],
        flounderExtraBindings = [("octopus", ["rpcclient"])],	
        addLibraries = libDeps ["trace", "skb", "vfs_ramfs", "lwip", "octopus", "linenoise", 							"posixcompat" ],
        flounderBindings = [ "pixels" ],
        architectures = [  "armv7", "armv8" ]              
    }
    \end{lstlisting}
    
    其中addLibraries定义了fish程序需要链接的库文件。

    影响最终生成文件大小的只与应用程序包含的C文件以及引用的库有关。

    生成的Makefile对应的规则中,先由fish程序所有的c文件生成对应的.o文件(由Hakefile中的cFiles字段定义)。在生成可执行文件时,把前一步生成的所有.o文件以及addLibraries字段中定义的库所对应的.a文件作为gcc的输入文件。然后就可以生成可执行文件了。

    addLibraries字段与裁剪的关系不大,若此处定义的库多余实际所需要的库,也并不会把多余的库的代码实际链接进目标程序。此处只是定义了依赖关系,当被依赖的库的代码改变时,将会使用改动后的库重新链接该应用程序。

    应用程序的C文件会调用一些库函数,GCC在链接所有输入文件时,只有被调用的库函数所在的C文件所生成的.o文件才会被实际链接到最终生成的文件中。因此,生成的可执行文件中包含的都是有用的代码,是裁剪之后的结果。
    
    \section{操作系统镜像的生成}
    
    操作系统镜像包含了boot driver, cpu driver以及Monitor, memory server, fish等一系列可执行文件。镜像是由arm\_bootimage工具生成的,该工具会读取指定的menu.lst文件。Menu.lst文件的格式如下所示。
    
    \begin{lstlisting}
    title	Barrelfish

    #root	(nd)
    kernel	/armv7/sbin/cpu_a15ve loglevel=3 periphbase=0x2c000000 consolePort=0
    module	/armv7/sbin/cpu_a15ve
    module	/armv7/sbin/init
    
    # Domains spawned by init
    module	/armv7/sbin/mem_serv
    module	/armv7/sbin/monitor
    
    # Special boot time domains spawned by monitor
    module /armv7/sbin/ramfsd boot
    module /armv7/sbin/skb boot
    modulenounzip /eclipseclp_ramfs.cpio.gz nospawn
    modulenounzip /skb_ramfs.cpio.gz nospawn
    module /armv7/sbin/kaluga boot add_device_db=plat_VE_A15x4
    module /armv7/sbin/spawnd boot
    module /armv7/sbin/startd boot
    
    # Device drivers
    # module /armv7/sbin/serial_pl011 auto
    module /armv7/sbin/serial_kernel irq=37
    module /armv7/sbin/corectrl auto
    
    # General user domains
    module /armv7/sbin/angler serial0.terminal dumb
    module /armv7/sbin/fish nospawn
    
    module /armv7/sbin/memtest
    
    # gem5 simulates 512MB of RAM starting at 0x80000000
    #        start       size       id
    mmap map 0x00000000  0x80000000 13 # Device region
    mmap map 0x80000000  0x20000000 1
    \end{lstlisting}
    
    该文件中定义了所有需要被打包到镜像中的内核以及用户空间程序。当不需要某项功能时,只需要在menu.lst文件中将相应的声明项去掉即可。

    因此,通过对menu.lst文件的修改,可以实现模块级别的裁剪。
    
    \chapter{中断分发}
    
    \section{ARMv7-a 中断处理}
    
    在中断控制器的寄存器中有一个名字为Interrupt Processor Targets Registers的寄存器,其负责指定每一个中断对应的Target,及谁来负责这个中断的处理。该寄存器可以指定一个Core,也可以指定多个Core,例如字段0b011表示由Core 0和Core 1来响应该中断。
    
    \textbf{当Target寄存器只指定单个Core响应中断时,中断被转发到该Core处理。}
    
    当Target寄存器只指定多个Core响应中断时,ARM GICv2中规定了中断的处理模式有两种,一种是1-N模式,第二种是N-N模式。
    
    \begin{description}
        \item[1-N模式] \textcolor{red}{Only one processor handles this interrupt.} The system must implement a mechanism to determine which processor handles an interrupt that is programmed to target more than one processor.
        \item[N-N模式] \textcolor{red}{All processors receive the interrupt independently.} When a processor acknowledges the interrupt, the interrupt pending state is cleared only for that processor. The interrupt remains pending for the other processors.
    \end{description}
    
    \section{Barrelfish中断处理}
    
    \subsection{中断控制器驱动}
    
    \begin{lstlisting}[language=C]
    void gic_enable_interrupt(uint32_t int_id, uint8_t cpu_targets, uint16_t prio,
                              bool edge_triggered, bool one_to_n)
    \end{lstlisting}
    
    Barrelfish在使能一个中断的接口函数上给出了cpu\_targets和one\_to\_n,前者指定响应的Core,后者表明是1-N还是N-N。所以在驱动实现上,对于我们组会上关心的问题是功能完备的。
    
    \subsection{中断系统调用}
   
    Barrelfish给出了irq\_table\_set的系统调用,在实现上我们找出中断控制器的设置部分。
    \begin{lstlisting}
    gic_enable_interrupt(nidt, BIT(my_core_id), 0,
                         GIC_IRQ_EDGE_TRIGGERED, GIC_IRQ_N_TO_N);
    \end{lstlisting}
    可以看到只有简单的一句话,含义是使能nidt给出的中断,由my\_core\_id指定的Core负责将该中断转发给用户驱动,即在哪注册,在哪处理。GIC\_IRQ\_EDGE\_TRIGGERED指定中断触发方式为边缘触发。并指定多Target时的处理模式为N-N。
    
    .
    
    .
    
    .
         
    以上就是Barrelfish现有的IRQ分发机制。如果要实现复杂一点的机制可以在中断系统调用上实现。
    
    
\end{document}